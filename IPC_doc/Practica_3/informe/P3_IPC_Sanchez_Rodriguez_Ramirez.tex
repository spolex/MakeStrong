\documentclass[a4paper]{article}
\usepackage[T1]{fontenc}
\usepackage[utf8]{inputenc}
\usepackage{lmodern}
\usepackage{graphicx}
\usepackage[left=2.5cm,right=2.5cm,top=3cm,bottom=3cm]{geometry}
\usepackage{eurosym}
\usepackage{fancyhdr}%encabezado y pie de página
\usepackage[colorlinks=true, linkcolor=black, urlcolor=blue]{hyperref}
\setcounter{secnumdepth}{5}
\usepackage[spanish]{babel}
\setcounter{tocdepth}{5}
\usepackage{colortbl}%para colorear tablas
\usepackage{tabularx}
\usepackage{placeins}%para poner barrera y no pasen de secciones los elementos flotantes
%\usepackage{wasysym} %para poner símbolos
\usepackage{bbding} %para poner símbolos



%para el mapa mental
\usepackage{tikz}
\usetikzlibrary{mindmap,trees}
\usepackage{verbatim}


\date{}
\author{D. Ramirez Ambrosi \\ J. I. Sánchez Méndez \\ J. Rodríguez Azpeleta}
\title{\begin{center}
\textbf{\Huge{Make Yourself Strong}} \\ Prototipo digital  \\Proyecto de la asignatura Interacción Persona Computador \\ \Huge{Grupo 10}
\end{center}}
\date{\today}


\pagestyle{fancy}
\rhead{
\textbf{Make Yourself Strong} \hfill \textbf{Fecha:} \date{\today}
}

\lhead{}

%Separación entre párrafos
\setlength{\parskip}{3mm}

%colores
\definecolor{verde}{RGB}{127,255,0}%color para la barra de titulo
\definecolor{rojo}{RGB}{255,0,0}%color para características
\renewcommand\listfigurename{\centering LISTA DE FIGURAS}

\begin{document}
\maketitle

\thispagestyle{empty}%para evitar enumeración de la página de la portada y del índice
\newpage
\tableofcontents%índice
\thispagestyle{empty}
\newpage



%lista de figuras 
%\renewcommand\listfigurename{\centering LISTA DE FIGURAS}
%\listoffigures
%\clearpage

%Lista de tablas
%\renewcommand{\listtablename}{\centering ÍNDICE DE TABLAS} %Para cambiar el índice de las tablas
%\listoftables
%\thispagestyle{empty}
%\newpage

\setcounter{page}{1}%Para reiniciar el contador de páginas en la página deseada


\section{Tareas realizables}

Las tareas implementadas en el \textbf{prototipo digital único} entregado son las presentadas a continuación. El prototipo se ha creado teniendo en cuenta que \textbf{hay un usuario logueado} en el sistema, \textbf{salvo en una} de las funcionalidades, donde se ha sido más conveniente mostrarla como si no lo estuviera:

\begin{itemize}
	\item   \textbf{Añadir un entrenamiento}: en este caso, ya que hay ya un entrenamiento introducido en la fecha indicada, se trata de \textbf{modificar} un entrenamiento. Sin embargo, esto nos sirve para \textbf{mostrar avisos} al estar un entrenamiento registrado y mostrar los datos ya almacenados para su edición. Así mismo, esta función implementa los \textbf{mensajes emergentes de ayuda} en las señales indicadas.
	
	\item   \textbf{Consultar a un experto}: se implementa la interacción propia del \textbf{envío de mensajes a un experto} que el usuario elija. En este caso \textbf{la interacción se ha preparado para un usuario no registrado} en la aplicación, de forma que se vea claramente que éstos usuarios también pueden probar los servicios.
	
	\item   \textbf{Búsqueda de gimnasios}: cualquier usuario puede realizar una búsqueda de sus gimnasios próximos.
	
	\item   \textbf{Obtención de entrenamiento personalizado}: se ha implementado de forma que los usuarios no registrados puedan elegir sus parámetros más adecuados y se muestre el entrenamiento más conveniente.
	
	\item   \textbf{Redacción de un artículo}: se permite acceder a la pantalla de redacción y, tras el ``envío'', mostrar los errores que pudieran darse.
	
	\item   \textbf{Navegación}: se han creado diferentes páginas generales y de la sección personal del usuario de forma que \textbf{se pueda navegar} por las mismas y \textbf{observar los contenidos} disponibles en ellas.
\end{itemize}

\section{Diferencias entre los prototipos}

A continuación se exponen los cambios introducidos en la versión 2 del prototipo sobre la versión 1:

\begin{itemize}
	\item   Iconos. Se han añadido iconos a los mensajes de la página indicando si muestran información o una advertencia. Así mismo, se han añadido iconos de ayuda a aquellos elementos que resultan de mayor complejidad y que creemos que los usuarios necesitarán una explicación para entenderlos. Estos iconos solo implementan la ayuda a la hora de añadir un entrenamiento.
	
	\item   Cambio en el formulario de contribución. Se ha añadido la sección de "introducción" al artículo que será redactado. Esta información será la que se muestre a la hora de listar las noticias. Así mismo, se permite la subida de una imagen para el artículo.
	
	\item   A la hora de buscar los gimnasios, se muestran directamente los resultados a partir de la ubicación del usuario. El usuario sigue teniendo la capacidad de buscar un lugar concreto de su elección.
	
	\item   Se permite filtrar los expertos por áreas concretas.
	
	\item   Se ha modificado la visualización del listado de planes de entrenamiento, permitiendo filtrarlos mediante un objetivo concreto entre los definidos.
	
	\item   Más cambios relacionados con la inclusión de imágenes se han incluido en las páginas de "guía de inicio", selección de parámetros del entrenamiento personalizado y el entrenamiento personalizado en sí.
	
	\item   Se ha añadido en preferencias la posibilidad de seleccionar las unidades de medida por parte del usuario y la posibilidad de eliminar la cuenta. Se ha añadido en la pantalla de preguntas frecuencias el acceso al borrado de la cuenta.
	
	\item   A la hora de visualizar una noticia, se han añadido los artículos relacionados con la misma.
	
	%Cambios añadidos a la version 2
	
	\item   Se han alineado a la izquierda las etiquetas de los apartados del menú de la sección personal del usuario.
	
	\item   Se ha añadido la posibilidad de saltar a una página concreta en las pantallas de visualización de listados. Estas páginas son el listado de noticias, el listado de planes de entrenamiento y el listado de expertos.
	
	\item   Rediseño de la función de añadir entrenamiento en forma de asistente. 
\end{itemize}


\section{Encuesta}
 Para la realización de esta práctica se ha creado una encuesta con el fin de identificar las funciones y apartados de mayor importancia para usuarios potenciales de la aplicación. Con ello, se ha creado una estructuración de menús teniendo en cuenta las opiniones recogidas.
 
 A continuación se incluye la encuesta realizada y, posteriormente, los resultados de la misma.
 
 \subsection{Anexo: encuesta}
 
 \includegraphics[width=\textwidth, page=1]{./figuras/encuesta.pdf}
 \includegraphics[width=\textwidth, page=2]{./figuras/encuesta.pdf}
 \includegraphics[width=\textwidth, page=3]{./figuras/encuesta.pdf}
 
 \subsection{Resultados de la encuesta}

\begin{figure}[!h]
\centering
 \includegraphics[angle=90,height=0.95\textheight,clip=true,trim=1cm 7cm 2cm 1cm]{./figuras/resultado-encuesta.pdf}
 \end{figure}
 
 
\end{document}