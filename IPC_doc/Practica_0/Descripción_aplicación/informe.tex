\documentclass[a4paper]{article}
\usepackage[T1]{fontenc}
\usepackage[utf8]{inputenc}
\usepackage{lmodern}
\usepackage{graphicx}
\usepackage[left=2.5cm,right=2.5cm,top=3cm,bottom=3cm]{geometry}
\usepackage{eurosym}
\usepackage{fancyhdr}%encabezado y pie de página
\usepackage[colorlinks=true, linkcolor=black, urlcolor=blue]{hyperref}
\setcounter{secnumdepth}{5}
\usepackage[spanish]{babel}
\setcounter{tocdepth}{5}
\usepackage{colortbl}%para colorear tablas
\usepackage{tabularx}
\usepackage{placeins}%para poner barrera y no pasen de secciones los elemntos flotantes
%\usepackage{wasysym} %para poner símbolos
\usepackage{bbding} %para poner símbolos



%para el mapa mental
\usepackage{tikz}
\usetikzlibrary{mindmap,trees}
\usepackage{verbatim}


\date{}
\author{D. Ramirez Ambrosi \\ J. I. Sánchez Méndez \\ J. Rodríguez Azpeleta}
\title{\begin{center}
\textbf{\Huge{Make Yourself Strong}} \\ Descripción de la aplicación \\ Fase de prototipado \\Proyecto de la asignatura Interacción Persona Computador \\ \Huge{Grupo 10}
\end{center}}
\date{\today}


\pagestyle{fancy}
\rhead{
\textbf{Make Yourself Strong} \hfill \textbf{Fecha:} \date{\today}
}

\lhead{}

%Separación entre párrafos
\setlength{\parskip}{3mm}

%colores
\definecolor{verde}{RGB}{127,255,0}%color para la barra de titulo
\definecolor{rojo}{RGB}{255,0,0}%color para características
\renewcommand\listfigurename{\centering LISTA DE FIGURAS}

\begin{document}
\maketitle

\thispagestyle{empty}%para evitar enumeración de la página de la portada y del índice
\newpage
\tableofcontents%índice
\thispagestyle{empty}
\newpage



%lista de figuras 
%\renewcommand\listfigurename{\centering LISTA DE FIGURAS}
%\listoffigures
%\clearpage

%Lista de tablas
%\renewcommand{\listtablename}{\centering ÍNDICE DE TABLAS} %Para cambiar el índice de las tablas
%\listoftables
%\thispagestyle{empty}
%\newpage

\setcounter{page}{1}%Para reinizar el contador de páginas en la página deseada

\section{Descripción de la aplicación}

Inicialmente \textbf{Make Yourself Strong} es una aplicación mostrada como una \textbf{página web} accesible desde cualquier navegador. Posteriormente se baraja la posibilidad de ofrecer una aplicación móvil que facilite el acceso a la misma desde terminales móviles.

\subsection{Público objetivo}
Esta aplicación va dirigida a todo aquel que quiera adquirir una base para poder aumentar de forma progresiva la fuerza. Con ella se pretende abarcar la segmentación de usuarios que:

\begin{itemize}
	\item	Deseen mejorar su forma física a través del ejercicio y una correcta nutrición que carecen de los recursos necesarios para contratar un profesional
	\item	Desean iniciarse sin la necesidad de pagar las tarifas de un entrenador profesional.
\end{itemize}


\subsection{Objetivos de la aplicación}

\begin{itemize}
	\item	Visualización de \textbf{noticias}. Visualización en lista ordenada de forma cronológica (más recientes primero). Posibilidad de comentar una noticia o compartirla en redes sociales.
	
	\item	Incorporación de \textbf{páginas de consulta} con diversos contenidos: rutinas de entrenamiento, nutrición, consejos...
	
	\item	Gestionar distintos perfiles de usuarios. La funcionalidad ofrecida será distinta según el tipo de perfil usado. Se pueden distirguir los perfiles: administrador, entrenador, usuario estándar y usuario premium.
	
	\item	Ofrecer un servicio de \textbf{entrenador virtual} que ayude al usuario de forma personal. El servicio será colaborativo, los usuarios con perfil entrenador guiarán a los usuarios premium que lo soliciten.
	
	\item	Facilitar la progresión y el seguimiento de los objetivos del usuario mediante \textbf{herramientas de registro} de estado físico y almacenamiento de entrenamientos realizados. Así mismo, se podrá consultar la información propia almacenada, \textbf{historial}, y el progreso mediante gráficos.
	
	\item	Permitir la \textbf{búsqueda de los gimnasios y tiendas} de nutrición más próximos al usuario en base al código postal.
\end{itemize}

En resumen facilitar el acceso gratuito a recursos generales de orientación y permitir la consulta a expertos en nutrición y preparación física de forma personalizadaa para cada usuario.

\section{Funcionalidad por usuario}

%Tabla 



\begin{center}
 
\begin{tabularx}{14cm}{|X||c|c|c|c|c|}
\hline
\textbf{Funcionalidad} & No registrado & Estándar & Premium & Entrenador & Administrador\\
\hline
Consultar noticias & \textcolor{verde}{\Checkmark}	& \textcolor{verde}{\Checkmark}	& \textcolor{verde}{\Checkmark}	& \textcolor{verde}{\Checkmark}	& \textcolor{verde}{\Checkmark}	\\
\hline
Comentar noticias & \textcolor{red}{\XSolidBrush}	& \textcolor{verde}{\Checkmark}	& \textcolor{verde}{\Checkmark}	& \textcolor{verde}{\Checkmark}	& \textcolor{verde}{\Checkmark}	\\
\hline
Añadir noticias & \textcolor{red}{\XSolidBrush}	& \textcolor{verde}{\Checkmark}	& \textcolor{verde}{\Checkmark}	& \textcolor{verde}{\Checkmark}	&  \textcolor{verde}{\Checkmark}	\\
\hline
Consultar gimnasios y tiendas cercanas & \textcolor{verde}{\Checkmark}	& \textcolor{verde}{\Checkmark}	& \textcolor{verde}{\Checkmark}	& \textcolor{verde}{\Checkmark}	& \textcolor{verde}{\Checkmark}	\\
\hline
Consultar entrenamientos & \textcolor{red}{\XSolidBrush}	& \textcolor{verde}{\Checkmark}	& \textcolor{verde}{\Checkmark}	& \textcolor{verde}{\Checkmark}	& \textcolor{verde}{\Checkmark}	\\
\hline
Consultas de entrenamientos personalizados  & \textcolor{red}{\XSolidBrush}	& \textcolor{red}{\XSolidBrush}	& \textcolor{verde}{\Checkmark}	& \textcolor{verde}{\Checkmark}	& \textcolor{verde}{\Checkmark}	\\
\hline
Preguntar a expertos & \textcolor{red}{\XSolidBrush}	& \textcolor{red}{\XSolidBrush}	& \textcolor{verde}{\Checkmark}	& 	& 	\\
\hline
Aconsejar entrenamientos & \textcolor{red}{\XSolidBrush}	& \textcolor{red}{\XSolidBrush}	& \textcolor{red}{\XSolidBrush}	& \textcolor{verde}{\Checkmark}	&	\\
\hline
Mantener histórico de entrenamientos & \textcolor{red}{\XSolidBrush}	& \textcolor{verde}{\Checkmark}	& \textcolor{verde}{\Checkmark}	& \textcolor{verde}{\Checkmark}	& \textcolor{verde}{\Checkmark}	\\
\hline
Ver estadísticas y progreso  & \textcolor{red}{\XSolidBrush}	& \textcolor{red}{\XSolidBrush}	& \textcolor{verde}{\Checkmark}	& \textcolor{verde}{\Checkmark}	& \textcolor{verde}{\Checkmark}	\\
\hline
Dar de alta entrenadores  & 	&	&	&	& \textcolor{verde}{\Checkmark}	\\
\hline
Administrar usuarios  &	&	&	&	& \textcolor{verde}{\Checkmark}	\\
\hline

\end{tabularx}

\end{center}

\section{Prototipo de la aplicación}

En el prototipo se considera el perfil de \textbf{usuario Premium}. La vista y opciones disponibles están preparadas para dicho tipo de usuario. Las funcionalidades cubiertas por el prototipo son:

	\begin{itemize}
		\item	Visualización de la sección de inicio y noticias de la página.
		\item	Visualización y navegación a través de 4 de los 5 apartados del perfil personal: Inicio del usuario, objetivos, evolución y  entrenamientos.
		\item	Proceso de registro de entrenamientos realizados por el usuario.
	\end{itemize}
	

\section{Test del prototipo}

\subsection{Situación a evaluar}

El usuario posee el perfil de usuario premium. El registro acaba de ser realizado y no hay datos almacenados sobre él (a parte de la propia información introducida en el registro).

El usuario tiene la posibilidad de navegar por las paginas de inicio y noticias, navegar y consultar la informació que tiene (o podría tener) disponible en su perfil y realizar el proceso de registro de un entrenamiento.

Se realizará una explicación de aquellas páginas o enlaces que el usuario quiera consultar y no estén disponibles al final de la prueba del prototipo.

\subsection{Prueba de la aplicación}

En la siguiente carpeta de Dropbox se incluye la prueba del prototipo en vídeo:

\url{https://www.dropbox.com/sh/6x8n5tdvvnwp35j/AADoKoAdi63DITYBnyizTVEka?dl=0}

\subsection{Errores y áreas de mejora localizados}

\subsubsection*{Estructura de la página}

La página sigue una estructuración correcta de los elementos, en cualquier caso el ajuste de los mismos mejorará con el prototipo digital. Los elementos que encontramos son:

	\begin{itemize}
		\item	\textbf{Cabecera}: contiene el título que se va actualizado según la página o conjunto de páginas visitadas y la barra de navegación, con diferentes campos que despliegan menús al pasar el cursor.
		\item	\textbf{Pie de página}: actualmente contendrá el mapa de la página web.
		\item	\textbf{Publicidad}: barra destinada a mostrar anuncios, en un principio fijos y, opcionalmente, específicos para cada usuario. Se situa en la parte derecha de la página debajo de la barra de navegación, es estrecha de aproximadamente un cuarto del ancho de la página.
		\item	\textbf{Sección principal}: es la parte donde se muestra la información correspondiente a cada página, es el elemento que no se mantiene de pantalla a pantalla. Ocupa la mayor parte de la pantalla (aproximadamente tres cuartos del ancho total) bajo la cabecera.
	\end{itemize}

En cuanto al contenido de esta estructura, concretamente la barra de menús, se deben modificar los elementos mostrados de formaa que se muestren aquellas opciones que van a ser más utilizadas por el usuario. Se debe cambiar:

	\begin{itemize}
		\item	El \textbf{menú de entrenamientos}. Eliminar entradas mostrando únicamente las necesarias. En este sentido sobra cada objetivo de entrenamiento específico, en su lugar se debería mostrar la opción "objetivos" que nos redireccionará a una página donde se muestren los diferentes posibles.
		\item	El\textbf{ menú del usuario}. En lugar de mostrar únicamente el nombre como enlace, se debería mostrar algo como "Accede a tu perfil *nombre usuario*", de forma que quede \textbf{más claro el acceso a su página personal}. Así mismo, en este mismo menú se deberían \textbf{incluir más funcionalidades} que no aparecen como "Añadir entrenamiento", que sería una función de uso habitual para registrar su actividad deportiva, o la de "actualizar objetivos", donde el usuario registra su estado físico a lo largo del tiempo actualizando datos como el peso actual.
	\end{itemize}
	

\subsubsection*{Área personal del usuario}

El menú izquierdo que aparece en la sección estructura la navegación por las diferentes opciones del perfil, separando adecuadamente la información. Aspecto a mejorar en este menú sería incluir iconos junto al nombre del menú.

Algunos defectos más concretos son:

	\begin{itemize}
		\item	En la pantalla de objetivos, incluir un botón aceptar para realizar la estimación del consumo de energía y nutrientes estimado, de forma que siga la tendencia de otros formularios realizados.
		\item	Mostrar botones de ayuda, junto a los accesos a las funciones o en una pantalla concreta dentro de una transacción, de forma que se explique la funcionalidad de dicha función o el paso siguienete que debe realizar para avanzar en la transacción concreta.
		\item	En general, mostrar más mensajes que indiquen al usuario si ha realizado cierta acción bien o debe corregir algún parámetro.
	\end{itemize}

\subsubsection*{Aspecto general}

Tratándose del prototipo en papel, no ha sido un aspecto al que le hayamos prestado gran atención. En algunos casos hemos utilizado colores símplemente para resaltar cierta opción o menú seleccionado.

El uso de colores y de tipos de letra convenientes se contemplará al desarrollar el prototipo digital.

En cuanto a la estructura, como ya se ha comentado previamente, y la disposición de los elementos (en la mayoría de las páginas prototipadas) sí que se ha conseguido una disposición adecuada de forma que se facilite la navegación al lector.


\end{document}
