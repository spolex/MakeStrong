\documentclass[a4paper]{article}
\usepackage[T1]{fontenc}
\usepackage[utf8]{inputenc}
\usepackage{lmodern}
\usepackage{graphicx}
\usepackage[left=2.5cm,right=2.5cm,top=3cm,bottom=3cm]{geometry}
\usepackage{eurosym}
\usepackage{fancyhdr}%encabezado y pie de página
\usepackage[colorlinks=true, linkcolor=black, urlcolor=blue]{hyperref}
\setcounter{secnumdepth}{5}
\usepackage[spanish]{babel}
\setcounter{tocdepth}{5}
\usepackage{colortbl}%para colorear tablas
\usepackage{tabularx}
\usepackage{placeins}%para poner barrera y no pasen de secciones los elemntos flotantes


%para el mapa mental
\usepackage{tikz}
\usetikzlibrary{mindmap,trees}
\usepackage{verbatim}


\date{}
\author{D. Ramirez Ambrosi \\ J. I. Sánchez Méndez \\ J. Rodríguez Azpeleta}
\title{\begin{center}
\textbf{\Huge{Make Yourself Strong}} \\ Descripción de la aplicación \\ Fase de prototipado \\Proyecto de la asignatura Interacción Persona Computador
\end{center}}
\date{\today}


\pagestyle{fancy}
\rhead{
\textbf{Make Yourself Strong} \hfill \textbf{Fecha:} \date{\today}
}

\lhead{}

%Separación entre párrafos
\setlength{\parskip}{3mm}

%colores
\definecolor{azul}{RGB}{0,240,255}%color para la barra de titulo
\definecolor{amarillo}{RGB}{255,240,0}%color para características
\renewcommand\listfigurename{\centering LISTA DE FIGURAS}

\begin{document}
\maketitle

\thispagestyle{empty}%para evitar enumeración de la página de la portada y del índice
\newpage
\tableofcontents%índice
\thispagestyle{empty}
\newpage



%lista de figuras 
%\renewcommand\listfigurename{\centering LISTA DE FIGURAS}
%\listoffigures
%\clearpage

%Lista de tablas
%\renewcommand{\listtablename}{\centering ÍNDICE DE TABLAS} %Para cambiar el índice de las tablas
%\listoftables
%\thispagestyle{empty}
%\newpage

\setcounter{page}{1}%Para reinizar el contador de páginas en la página deseada

\section{Descripción de la aplicación}

\subsection{Público objetivo}

\subsection{Objetivos de la aplicación}

\begin{itemize}
	\item	Visualización de \textbf{noticias}. Visualización en lista ordenada de forma cronológica (más recientes primero), ¿con posibiidad de buscar?. Posibilidad de comentar una noticia o compartirla en redes ssociales.
	
	\item	Incorporación de una \textbf{Wiki de consulta} con diversos contenidos: rutinas de entrenamiento, nutrición, consejos...
	
	\item	Ofrecer una \textbf{tienda virtual} para la venta de suplementos.
	
	\item	Ofrecer un servicio de \textbf{entrenador virtual} que ayude al usuario de forma personal.
\end{itemize}

\section{Test del prototipo}

\subsection{Patrones de error detectados}


\end{document}