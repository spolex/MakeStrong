\documentclass[a4paper]{article}
\usepackage[T1]{fontenc}
\usepackage[utf8]{inputenc}
\usepackage{lmodern}
\usepackage{graphicx}
\usepackage[left=2.5cm,right=2.5cm,top=3cm,bottom=3cm]{geometry}
\usepackage{eurosym}
\usepackage{fancyhdr}%encabezado y pie de página
\usepackage[colorlinks=true, linkcolor=black, urlcolor=blue]{hyperref}
\setcounter{secnumdepth}{5}
\usepackage[spanish]{babel}
\setcounter{tocdepth}{5}
\usepackage{colortbl}%para colorear tablas
\usepackage{tabularx}
\usepackage{placeins}%para poner barrera y no pasen de secciones los elementos flotantes
%\usepackage{wasysym} %para poner símbolos
\usepackage{bbding} %para poner símbolos



%para el mapa mental
\usepackage{tikz}
\usetikzlibrary{mindmap,trees}
\usepackage{verbatim}


\date{}
\author{D. Ramirez Ambrosi \\ J. I. Sánchez Méndez \\ J. Rodríguez Azpeleta}
\title{\begin{center}
\textbf{\Huge{Make Yourself Strong}} \\ Análisis de necesidades  \\Proyecto de la asignatura Interacción Persona Computador \\ \Huge{Grupo 10}
\end{center}}
\date{\today}


\pagestyle{fancy}
\rhead{
\textbf{Make Yourself Strong} \hfill \textbf{Fecha:} \date{\today}
}

\lhead{}

%Separación entre párrafos
\setlength{\parskip}{3mm}

%colores
\definecolor{verde}{RGB}{127,255,0}%color para la barra de titulo
\definecolor{rojo}{RGB}{255,0,0}%color para características
\renewcommand\listfigurename{\centering LISTA DE FIGURAS}

\begin{document}
\maketitle

\thispagestyle{empty}%para evitar enumeración de la página de la portada y del índice
\newpage
\tableofcontents%índice
\thispagestyle{empty}
\newpage



%lista de figuras 
%\renewcommand\listfigurename{\centering LISTA DE FIGURAS}
%\listoffigures
%\clearpage

%Lista de tablas
%\renewcommand{\listtablename}{\centering ÍNDICE DE TABLAS} %Para cambiar el índice de las tablas
%\listoftables
%\thispagestyle{empty}
%\newpage

\setcounter{page}{1}%Para reiniciar el contador de páginas en la página deseada


\section{Introducción}

La prueba está basada en el uso de la aplicación \textbf{MyFitnessPAL} \footnote{\url{http://www.myfitnesspal.com.mx}}. Esta aplicación está centrada en la cuenta de calorías de un usuario. En base a parámetros - como la altura, peso y nivel de actividad física - y objetivos - como mantener o variar el peso - muestra una estimación de las calorías que el usuario debe consumir al día así como la cantidad de macronutrientes (hidratos de carbono, grasas y proteínas) que debe consumir para alcanzar los objetivos deseados.

Los usuarios pueden realizar el seguimiento calórico mediante el registro de los alimentos consumidos mediante búsquedas en la base de datos propia. También tienen la posibilidad de añadir un alimento en caso de no encontrarlo en dicha base de datos.

También tienen la posibilidad de registrar las calorías consumidas por ejercicio físico. Como en el caso anterior, se puede buscar el ejercicio en la base de datos o añadirlo en caso de que no se encuentre.

Los usuarios pueden realizar su propio seguimiento mediante la consulta de estadísticas gráficas.

\section{Observación de participantes}

	\subsection{Jonathan Castro}
	
	Estudiante de Ingeniería informática, práctica deporte de forma regular.
	
		\subsubsection*{Registro de usuario}
		
		Llevado a cabo sin mayores dificultades. Resultó molesto que se pidiera información como la fecha exacta de nacimiento - Con el año para el cálculo de la edad serviría - así como el código postal y ciudad de residencia.
		
		Durante el registro se pregunta sobre los objetivos, las descripciones del nivel de actividad no resultan muy descriptivas. Además, no determinan de forma clara si en ese nivel de actividad se incluye el que se haga ejercicio regularmente o únicamente trata el nivel de actividad habitual referente a la profesión de la persona.
		
		\subsubsection*{Añadir un entrenamiento}
		
		Tras añadir un entrenamiento, se muestra en una tabla. Se da la opción de eliminarlo, aunque no se ve claramente está función ya que viene representada por un icono con una señal de prohibición. Un icono más representativo de la función eliminar evitaría confusiones.
		
		En el caso de añadir un entrenamiento de fuerza, solo se puede añadir para un entrenamiento determinado un número de series con el mismo peso y repeticiones para cada una de las mismas. En este caso se hubiera querido especificar el peso y repeticiones para cada serie concreta.
		
		Al contabilizar las calorías quemadas en el entrenamiento, se muestran solo las correspondientes al entrenamiento cardiovascular y no se tienen en cuenta las del entrenamiento de fuerza.
		
		Otro defecto fue que las estimaciones de las calorías no eran exactas en determinados entrenamientos.
		
		Una sugerencia en este apartado fue que se tuvieran en cuenta patologías que el usuario pudiera tener para que la aplicación pudiera indicar si se daban casos de sobreesfuerzo.
		
		\subsubsection*{Añadir alimentos}
		
		En general, ningún defecto en cuanto a la interacción. El problema viene de la estimación dada para algunos alimentos
		
		\subsubsection*{Comentarios y mejoras generales}
		
		Estimaciones, tanto en calorías de los entrenamientos, como de macronutrientes de los alimentos y estimación de objetivos.
		
		En cuanto a menús, los botones de sesión, menos destacados que el resto, quedan poco a la vista. El calendario para selección de fechas está en formato inglés, siendo domingo el día que se encuentra en primer lugar.
		
			\begin{figure}[!h]
				\centering
				\includegraphics[width=0.7\textwidth]{./figuras/jonat-1.jpg}
				\caption{Jonathan navegando por la aplicación}
			\end{figure} 
		
			\begin{figure}[!h]
				\centering
				\includegraphics[width=0.5\textwidth]{./figuras/jonat-2.jpg}
				\caption{Página principal}
			\end{figure} 
			
			\begin{figure}[!h]
				\centering
				\includegraphics[width=0.5\textwidth]{./figuras/jonat-3.jpg}
				\caption{Añadiendo un ejercicio de fuerza}
			\end{figure}
			
			\begin{figure}[!h]
				\centering
				\includegraphics[width=0.5\textwidth]{./figuras/jonat-4.jpg}
				\caption{Icono incorrecto para representar `eliminar''}
			\end{figure}
		
		\FloatBarrier
		
		\subsection{Marta Aguilera}
		
		Trabaja actualmente como desarrolladora de software.
		
			\subsubsection*{Registro de usuario}
			
			Algunos datos de los requeridos por la aplicación, como el código postal y la fecha exacta de nacimiento, deberían ser opcionales. En cuanto a la fecha de nacimiento, con saber el año para calcular edad sería más que suficiente como parámetro para la aplicación.
			
			El vocabulario usado para preguntar sobre la frecuencia con que se va a practicar ejercicio resulta poco conciso. La aplicación pregunta por el número de "rutinas por semana". Ha sido necesario aclarar que se trata de días por semana que se va a practicar ejercicio, ya que rutina se asocia a una serie de ejercicios que se practican en una sesión de entrenamiento y que esta misma rutina se puede repetir varios días.
			
			En cuanto a la función de invitar a amigos, no le resulta demasiado llamativa. Si que aprecia el mensaje en que la página asegura la privacidad del peso registrado y que éste no será visible a otros usuarios.
			
			En la siguiente pantalla de objetivo los números están muy bien, pero se echan en falta explicaciones o consejos como ``evita comer determinado tipo de alimentos''.
			
			\subsubsection*{Añadir un entrenamiento}
			
			La búsqueda de ejercicios debe ser realizada por el nombre específico del mismo. Por ejemplo, al buscar ``cinta'', no muestra ningún resultado. En su lugar hay que buscar ``correr'' donde uno de los resultados indica ``correr en el sitio'' que sería el correspondiente a usar la cinta de correr. En este sentido, la búsqueda de ejercicio debería ser "más abierta" permitiendo búsquedas por máquinas concretas o nombres más coloquiales que se pudieran al ejercicio en cuestión. Esto permitiría que la búsqueda fuera más intuitiva y no frustrante como ha resultado ligeramente para Marta.
			
			A destacar de forma positiva que los ejercicios seleccionados de forma reciente estén disponibles (bajo la barra de búsqueda) para seleccionarlos directamente sin necesidad de buscarlos de nuevo.
			
			En la pantalla general del entrenamiento destaca la función de ``herramientas rápidas'' que permiten copiar la lista de ejercicios desde una fecha concreta o copiar la lista del día actual a una fecha concreta. En cambio, el icono destinado a simbolizar la función de eliminar un entrenamiento de la lista no resulta representativo, ya que se trata de una señal de prohibición.
			
			Las listas de ``Cardiovascular'' y ``Fuerza'' están demasiado separadas entre ellas, además no se contabilizan las calorías quemadas en el entrenamiento de fuerza.
			
			Resulta de gran utilidad la función de ``Ver el informe completo (para imprimir)'' que permite ver el resumen del entrenamiento. En el se puede seleccionar un intervalo de fechas para las cuales se quiera practicar el entrenamiento creado. Está funcionalidad resultaría más útil antes, ya que permitiría aplicar el entrenamiento creado en los días deseados y sería más rápido que la opción de copiar a una fecha concreta que se ha visto antes. Un elemento que se hecha en falta en esta página de informe en un botón de vuelta atrás o de ``Terminado''.
			
			\subsubsection*{Añadir alimentos}
			
			Traducción incorrecta a diferentes idiomas, por ejemplo al añadir café, la cantidad se especificaba en número de ``cups''. En la búsqueda de alimentos resulta muy útil la posibilidad de hacer una búsqueda abierta como ``café'' y que se muestren varias opciones para seleccionar una concreta (un tipo de café determinado, una marca concreta...). Éste es el tipo de búsqueda que se echaba en falta a la hora de buscar los entrenamientos. Además se aprecia que la cantidad de entradas de alimentos es muy grande, hay gran cantidad de datos en la base de datos de la aplicación.
			
			Otro aspecto positivo es que la misma aplicación te avisa si sobrepasas los objetivos de algún elemento como las calorías, los azúcares, grasas... y también muestra mensajes gratificantes como "Si todos los días fueran como hoy, pesarías X en 5 semanas".
			
			En cuanto al agua consumida, sería mejor añadirla por comida en vez de globalmente la consumida en el día. Además, resulta poco gráfico la entrada de agua con el icono del vaso.
			
			\begin{figure}[!h]
				\centering
				\includegraphics[width=0.5\textwidth]{./figuras/marta1.jpg}
				\caption{Uso incorrecto de ``rutina'' para hacer referencia a días o sesión de entrenamiento}
			\end{figure}
			\begin{figure}[!h]
				\centering
				\includegraphics[width=0.7\textwidth]{./figuras/marta2.jpg}
				\caption{Mezcla de lenguajes ``cup'' en vez de ``taza''}
			\end{figure}
			\begin{figure}[!h]
				\centering
				\includegraphics[width=0.7\textwidth]{./figuras/marta3.jpg}
				\caption{Mensaje gratificante para el usuario}
			\end{figure}
			\begin{figure}[!h]
				\centering
				\includegraphics[width=0.8\textwidth]{./figuras/marta4.jpg}
				\caption{Importancia de la privacidad de los datos}
			\end{figure}
		\FloatBarrier
		
		\subsection{Andoni Martín}
		
			Estudiante de ingeniería informática y aficionado al trabajo hortelano.
		
			\subsubsection*{Registro de usuario}
			
			En el registro está marcada por defecto la casilla para que la aplicación envíe automáticamente boletines al correo, no resultaría molesto si la casilla estuviera desactivada por defecto. Además, no pide confirmación de la contraseña, es decir, no hay que escribirla dos veces para comprobar que se ha escrito correctamente. Esto podría hacer que el usuario no pudiera autenticarse en el sistema tras acabar el registro y tuviera que solicitar la recuperación o cambio de la contraseña.
			
			En cuanto a datos personales se repite la opinión de que pedir la fecha de nacimiento completa no resulta necesario (el año serviría) y que este dato junto con el código postal deberían ser opcionales.
			
			A la hora de indicar la cantidad de veces que se practica deporte ha sido necesario explicar que ``rutina'' en ese caso se refería a los días a la semana que se realizará actividad física. El lenguaje utilizado no es específico y aclaratorio.
			
			En cuanto a la pantalla de selección de objetivos, se debe especificar el peso actual y el deseado, pero posteriormente pregunta cuál es tu objetivo entre opciones de ``bajar X kilos a la semana'', ``mantener'' o ``subir X a la semana''. El problema radica en que puedes indicar bajar 10 kilos de peso al especificar el peso actual y el objetivo, y luego seleccionar como objetivo aumentar peso. No resulta lógica tal selección, es probable que a un usuario no se le ocurra realizar tal selección, pero la aplicación debería avisar de esta incongruencia en caso de que se diera.
			
			El resumen de objetivos que se muestra al finalizar el registro resulta útil.
			
			\subsubsection*{Añadir un entrenamiento}
			
			En la pantalla se hace la distinción entre entrenamiento ``cardiovascular'' y ``de fuerza'', Andoni sabía a qué se refería cada uno, pero otros usuario pueden necesitar de ayuda, una explicación en pantalla de cuál es la diferencia entre ambos y por qué están separados.
			
			A la hora de añadir un entrenamiento de tipo cardiovascular, se pregunta por el tiempo de duración de dicho ejercicio, pero no se especifican las unidades en que se mide el tiempo. En cuanto a la búsqueda, le gusta que aparezca ``pasear al perro'' y que conduzca a una única opción, aunque, en general, la búsqueda de ejercicios debe hacerse de forma específica con el nombre técnico del entrenamiento.
			
			Además de añadir los diferentes ejercicios al entrenamiento, también se permiten añadir notas sobre el mismo. Resulta molesto que no se pueda escribir directamente en el campo de texto, que sería lo más directo, y que haya que pulsar en un pequeño botón de editar.
			
			Otra cosa que resulta un tanto molesta es que el entrenamiento se guarde automáticamente a cada ejercicio que se introduce. En su lugar, estaría mejor un botón de ``Aceptar'' o ``Guardar entrenamiento'' y que el botón de ``Imprimir entrenamiento'' apareciera como secundario del formulario y no como único que es el caso actual.
			
			\subsubsection*{Añadir alimentos}
			
			A la hora de añadir alimentos, en el caso de algunos pedían cantidades usando ``poco'' como unidad. Podías haber comido un poco, dos pocos... pero ¿cuánto es un poco? En este sentido se deben especificar mejor las unidades de medida para los diferentes alimentos. Además, le ha resultado molesto que la búsqueda de un elemento determinado, por ejemplo ``zanahoria'', resultara en tantas posibilidades.
			
			Le ha parecido útil que el programa indique si el consumo de calorías totales se encuentra muy por debajo del objetivo total del día.
			
			En cuanto a la cuenta del agua, el vaso no le resulta una medida estándar, cada uno puede beber en un vaso más grande o más pequeño
			
			\subsubsection*{Comentarios y mejoras generales}
			
			En cuanto al diseño de la página, las acciones en algunos sitios se presentan como botones (en la página principal por ejemplo) y en otros casos como enlaces (en las ventanas de entrenamiento o alimentación, las funciones de añadir entrenamientos o alimentos respectivamente).
			
			En la pantalla principal no resultan claros los nombres de las funcionalidades dados en los botones. ``Añadir alimento'' enlaza con la página para añadir los alimentos de un día, pero ``Añadir'' a secas enlaza con añadir un entrenamiento. Debería ser ``añadir entrenamiento'' que especifica qué es lo que se va a añadir en ese botón.
			
			En esta pantalla se muestra también una barra que indica las calorías consumidas y restantes del día. Su diseño induce a pensar que se trata de un elemento deslizable que el usuario puede ajustar.
			
			\begin{figure}[!h]
				\centering
				\includegraphics[width=0.5\textwidth]{./figuras/andoni1.jpg}
				\caption{Falso slider, botón ``Añadir'' para añadir un entrenamiento y ``Añadir alimento'' especificado correctamente}
			\end{figure}
			\begin{figure}[!h]
				\centering
				\includegraphics[width=0.5\textwidth]{./figuras/andoni2.jpg}
				\caption{Unidades de medida del tiempo no especificadas}
			\end{figure}
			\begin{figure}[!h]
				\centering
				\includegraphics[width=0.7\textwidth]{./figuras/andoni3.jpg}
				\caption{No es posible escribir notas directamente sobre el espacio reservado}
			\end{figure}
			\begin{figure}[!h]
				\centering
				\includegraphics[width=0.5\textwidth]{./figuras/andoni4.jpg}
				\caption{Ración de ``un poco''}
			\end{figure}
			\begin{figure}[!h]
				\centering
				\includegraphics[width=0.7\textwidth]{./figuras/andoni5.jpg}
				\caption{Alerta por consumo insuficiente}
			\end{figure}
		\FloatBarrier
		

\section{Necesidades identificadas}

A continuación se detallan las diferentes necesidades detectadas en las diferentes observaciones.

	\subsection{Exactitud en las descripciones de elementos}
	
	Las descripciones que se asocien a diferentes elementos de una lista de opciones tienen que ser lo suficientemente claras para que el usuario no tenga ninguna duda de cuál es la que le conviene marcar.
	
	Este problema se da en el caso de la selección de la actividad diaria realizada donde se ejemplifica cada opción con una profesión pero no queda claro si se incluye la actividad física no asociada a la profesión que la persona pueda realizar.
	
	\subsection{Estimaciones más exactas}
	
	En general, ya sea al estimar los objetivos del usuario, las calorías y macronutrientes asociados a un alimento o las calorías consumidas por un determinado ejercicio, procurar, en la medida de lo posible, ofrecer valores más exactos al usuario.
	
	\subsection{Localización del usuario}
	
	En el caso del calendario, su formato era inglés. De cara a ofrecer un mejor servicio, se debería adaptar el calendario o incluso las unidades de medida al sistema utilizado en el lugar desde el que el usuario realiza la conexión.
	
	Otro problema estaba relacionado con la traducción. Ésta deberá estar lo más cuidada posible para que no se produzcan problemas o mal entendidos a usuarios de diferente habla.
	
	\subsection{Iconos representativos}
	
	En el caso de funciones básicas como eliminar un elemento, añadirlo o modificarlo, no se puede dar lugar a dudas de la acción representada por el icono en cuestión.
	
	\subsection{Registro de entrenamientos más preciso}
	
	Permitir especificar para un ejercicio de fuerza las diferentes series de forma individual, con un peso y repeticiones concretos para cada serie.
	
	\subsection{Registro de patologías}
	
	Permitir al usuario problemas que padezca (asma, problemas de corazón, daños musculares...), de forma que la aplicación avise en caso de sobre esfuerzo en base a los ejercicios añadidos a un entrenamiento.
	
	\subsection{Destacar los botones de sesión}
	
	Hacer que sean perfectamente visibles para que el usuario encuentre fácilmente, por ejemplo, el cierre de sesión.

	\subsection{Asegurar la privacidad de los datos}
	
	Se debe garantizar que los datos de los usuarios serán seguros ante cualquier ataque y que los mismos no serán accesibles. Se trata de garantizar que datos personales como la información de contacto (correo electrónico), información de localización (como el código postal) o simplemente el peso registrado del usuario no serán accedidos por terceras personas a no ser que el usuario lo permita.
	
	\subsection{Registro de datos no necesarios}
	
	Se ha dado el caso de pedir el código postal o fecha exacta de nacimiento del usuario a la hora de realizar el registro. Sería más adecuado que los usuarios tuvieran la posibilidad de elegir si introducirlos o no en el sistema. La aplicación no tiene necesidad de saber en qué área vive el usuario o el día y mes de nacimiento cuando, para el cálculo de las estimaciones, es suficiente con el año de nacimiento.
	
	\subsection{Uso de lenguaje específico}
	
	El vocabulario a utilizar debe ser específico del área en que se desarrolla la aplicación. Ya se han mostrado el ejemplo de ``rutina'' usado en lugar de ``días'' o ``sesión'' que encajarían mejor en el contexto en que se producía confusión.
	
	Otro caso ambiguo se encontraba en la página principal donde el botón ``Añadir'' se refería a añadir un entrenamiento.
	
	\subsection{Feedback y ayuda al usuario}
	
	Sobre todo para usuarios que utilizan una aplicación de este tipo por primera vez y no están familiarizados con los términos o la forma, correcta o no, de añadir entradas de ejercicios o comida, resulta muy importante que se le informe de cómo se desarrolla el proceso. Además, también hay que considerar informar al usuario sobre cómo desarrolla actividad, animarle.  A continuación repasamos varios ejemplos, estando los que hemos encontrado y han sido positivos, y los que se han echado en falta:
	
		\begin{itemize}
		\item	Tras terminar la pantalla de registro, se muestran los objetivos del usuario. Es una información muy útil y muy bien resumida y presentada. Usuarios con conocimiento en la materia que nos incumbe no tendrán problema en su interpretación, pero se debería ofrecer una opción que explique a usuarios principiantes lo que significan los diferentes elementos.
		
		\item	El uso de mensajes motivadores, concretamente al establecer los alimentos consumidos en el día y calcular el consumo calórico, animando al usuario a seguir así resulta agradable.
		
		\item	También el caso opuesto, estar muy por debajo o por encima del consumo calórico hace que la aplicación avise de tal hecho e indica al usuario los errores que tiene que corregir.
		
		\item	En el caso de añadir ejercicios a un entrenamiento, el programa distingue entre ejercicios de fuerza y cardiovasculares. Para un usuario sin experiencia sería necesario ofrecer una explicación de lo que es cada tipo de ejercicio y qué propósitos tiene.
		\end{itemize}
	
	\subsection{Facilitar tareas al usuario}
	
	Hacer las operaciones lo más fáciles posibles al usuario, intentando minimizar la repetición de acciones exactas. En esta línea, en la aplicación se han observado las herramientas rápidas para copiar el entrenamiento realizado o los alimentos consumidos de una fecha concreta a otra, de forma que si la entrada va a ser exactamente la misma o muy parecida no haya que empezar a, por ejemplo, añadir todos los ejercicios desde cero.
	
	Otra herramienta de utilidad ha sido la posibilidad de obtener una vista de impresión del entrenamiento, de forma que el formato se adecue a las características del papel y el usuario pueda obtener fácilmente una copia física de los datos del entrenamiento.
	
	También hay que tener en cuenta la posibilidad de acceder a un resumen de los objetivos propuestos para el usuario.
	
	Hacer que cuadros de entrada de texto sean directamente editables sin necesidad de tener que pulsar un botón concreto para su edición, ya que la primera es la opción más natural.
	
	\subsection{Opciones por defecto}
	
	Por ejemplo, en el formulario de registro se ofrece la posibilidad de recibir automáticamente un boletín de noticias al correo proporcionado. Esta opción viene marcada por defecto en el formulario y puede resultar un tanto molesta si no se desactiva.
	
	Aunque no se ha tratado en las observaciones, en el formulario de inicio de sesión también aparece marcada la casilla para recordar al usuario la próxima vez que vuelva a la página. Esta casilla aparece marcada por defecto y, si al usuario no le interesa, hay que desactivarla cada vez que se hace el login. Puede resultar perjudicial para el usuario en caso que acceda desde un ordenador ajeno.
	
	\subsection{Comprobación de errores}
	
	En el caso del registro, la contraseña se pide solo una vez, pero si el usuario la escribe mal, luego no podrá acceder. Es necesario pedir confirmación de la contraseña para evitar este tipo de problema.
	
	\subsection{Restringir opciones}
	
	Se deben utilizar para evitar contradicciones. En el caso de la definición de objetivos, el especificar un peso objetivo por debajo del actual, debería eliminar las opciones de ganar peso del desplegable concreto de los objetivos. De la misma forma se debería cumplir la inversa, con un peso objetivo por encima del actual y el caso de mantener el peso actual si los pesos objetivo y actual son iguales.
	
	\subsection{Búsquedas adecuadas a diferentes tipos de usuario}
	
	Se han dado dos casos opuestos, problemas por carencia de resultados y problemas por exceso de los mismos.
	
	Por defecto, en el caso de la búsqueda de ejercicios donde los nombres de los ejercicios resultaban muy específicos. Debería guardarse por cada ejercicio nombres más comunes que permitan realizar búsquedas más abiertas.
	
	En el caso de alimentos el exceso se debe a que una palabra de búsqueda puede dar lugar a muchos tipos de alimentos, al mismo alimento en muchas marcas diferentes o incluso a varias entradas de exactamente el mismo alimento (aunque con un ligero cambio de nombre). En este caso habría que limitar de alguna manera la búsqueda, aunque en este tema ha habido diferencias de opinión entre los diferentes participantes de la prueba.
	
	\subsection{Coherencia en el diseño de la aplicación}
	
	La página debe utilizar los mismos elementos en sus diferentes vistas. De este modo se ha observado que en la pantalla principal las opciones para acceder a las funciones de añadir entrenamientos o alimentos se dispone de botones. Sin embargo, una vez entramos en cualquiera de las dos, a la función de añadir un alimento o entrenamiento hay que acceder a través de un enlace contenido en texto, destacado eso sí pero texto. Ya que ambas funciones, añadir en la página principal o añadir en un entrenamiento, son similares, se debería utilizar en ambos sitios el mismo formato de botón para el acceso.
	
	\subsection{Especificación de unidades}
	
	Las unidades utilizadas en cada momento deben estar perfectamente bien indicadas y deben definir una cantidad concreta. Fallos concretos los encontramos a la hora de añadir un entrenamiento cardiovascular donde no se especifica que la unidad de medida en ese caso es el minuto. Otro caso se da en alimentos con porciones de ``un poco'' y otras medidas que pueden variar para los diferentes usuarios, ya se sabe que tomar un poco de algo depende de lo que le guste ese algo a cada usuario.
	
	De la misma forma, para especificar el agua consumida durante un día se indican el número de vasos que han sido consumidos, sin especificar en ningún sitio el tamaño del vaso.
	
	\subsection{Confirmación para el almacenamiento de datos}
	
	Hay que dejar al usuario que sea quien determine cuando se guardan los datos de un entrenamiento o comida en el sistema. En la aplicación testada se ha observado que los datos de un entrenamiento tras añadir un ejercicio son guardados automáticamente, careciendo la página de un botón dedicado a tal función. Debería modificarse este comportamiento en favor del control por parte del usuario.
	
	\subsection{Segmentación de usuarios}
	
	Tras observar participantes con distinta experiencia deportiva o nutricional, una buena solución sería que los usuarios estuvieran en grupos separados. De esta forma, la interfaz mostrada a cada uno variaría ligeramente y permitiría a usuarios expertos disponer de una interfaz ligeramente más compleja en cuanto a no mostrar ningún tipo de ayuda o mensaje aclaratorio. por otro lado, usuarios que precisaran de ayuda tendrían una interfaz algo más cargada a la vez que sencilla que guiara sus pasos el las diferentes partes de los procesos.
	
	\subsection{Personalización a nivel de usuario}
	
	Entre los tres participantes observados, se han dado diferencias en cuanto a cómo quería cada uno que se mostraran ciertas cosas. Por ejemplo las búsquedas de alimentos, en unos casos los resultados eran excesivos, en otros resultaba agradable que se mostraran tantos. En cuanto a búsquedas de entrenamiento, usuarios expertos no tendrán problema en encontrar los diferentes ejercicios, pero usuarios meno hábiles en la materia querrán optar por búsquedas más coloquiales.
	
	Esto lleva a la necesidad de recoger ciertos parámetros de personalización que se establecieran por defecto al registrar a un usuario pero que, posteriormente, pudieran ser modificados por el mismo y personalizados a los gustos y preferencias de cada uno.
	

\end{document}